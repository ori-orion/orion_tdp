\section*{HSR Software and External Devices}
\label{sec:annex-DSPL}
% In this section briefly describe the software and hardware of the robot
%Annex
%	Photo(s) of the robot - %TODO - Add more if space remains
%	Brief, compact list of the 3rd party robot’s software (e.g. include MoveIt/YOLO)
%	Brief, compact description of all external computing devices, if any
%	Brief, compact description of the robot’s hardware (OPL Only)
%	Please mark with an asterisk home-made software solutions
%	Annex should be appended after the References
%	There is no page limit for annex, but a maximum of one page is strongly encouraged


\setlength\intextsep{0pt}
\begin{wrapfigure}[10]{r}{0.3\textwidth}
	\centering
	\includegraphics[width=0.3\textwidth]{images/hsr.jpg}
	\caption{Team ORIon's HSR (``\textit{Bamm-Bamm}")}
	\label{fig:bamm-bamm}
\end{wrapfigure}

We use a standard DSPL HSR robot from \textit{Toyota}. No modifications have been applied.

\section*{Robot's Software Description}
% Please describe in this section the software you are using to control your robot. Consider the following example:

\textit{For our robot we are using the following software:}

\begin{itemize}
	\item Platform: Ubuntu 18.04, ROS Melodic (upgrade to Ubuntu 20.04 / ROS Noetic planned for 2022) 
	\item Navigation: STRANDS Navigation, ROS  \textit{move\_base}
	\item Semantic mapping: STRANDS Semantic Object Maps (SOMa), MongoDB
	\item Object recognition: YOLO object detector
	\item Arm control and gripper coordination: MoveIt
	\item Task-level planning: SMACH ROS library
	\item Grasp pose estimation: Grasp Pose Detection  (GPD)\footnote{https://github.com/atenpas/gpd}, Grasp Pose Generator (GPG)\footnote{https://github.com/atenpas/gpg}
	\item Speech recognition: PocketSphinx\footnote{https://github.com/bambocher/pocketsphinx-python} and Wavenet\footnote{https://github.com/buriburisuri/speech-to-text-wavenet} (offline speech recognition), Levenshtein\footnote{https://github.com/maxbachmann/Levenshtein} (semantic similarity checking), WordNet\footnote{https://www.nltk.org/howto/wordnet.html} (synonym checking) 
\end{itemize}

\section*{External Devices}
% Please describe in this section the external devices used by your robot. 

\textit{The HSR robot relies on the following external hardware:}

\begin{itemize}
	\item Alienware laptop in robot mount (for object recognition and pose estimation)
\end{itemize}

In 2022, we plan to transition to using the HSR GPU (NVIDIA Jetson) to process the object detection computational load instead of the Alienware laptop.

\section*{Cloud Services}
% Please describe in this section the Cloud Services and online software used by your robot.

\textit{The HSR connects the following cloud services:}
\begin{itemize}
	\item Speech recognition: All-purpose recogniser (Google API).
\end{itemize}