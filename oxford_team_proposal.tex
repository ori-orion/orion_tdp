%1234567890123456789012345678901234567890123456789012345678901234567890123456789
%         1         2         3         4         5         6         7        8

\documentclass[runningheads,a4paper]{llncs}
\usepackage{amssymb}
\setcounter{tocdepth}{3}
\usepackage{graphicx}
\usepackage{amssymb}
\usepackage[utf8]{inputenc}
\usepackage{url}
\usepackage{float}
\usepackage{amsmath}
\usepackage{graphicx}
\usepackage[tight]{subfigure}
\usepackage{wrapfig}

\usepackage{lipsum}
\newcommand{\BnL}[1][1em]{ \includegraphics[width=#1]{images/bnl.jpg} }

\begin{document}

\title{Lipsum 2017 Team Description Paper}

\author{Main-author \and Co-author \and Team Members }
\institute{Affiliation name and address, \\
\texttt{http://devoted-web-site.url}}
\maketitle


%%%%%%%%%%%%%%%%%%%%%%%%%%%%%%%%%%%%%%%%%%%%%%%%%%%%%%%%%%%%%%%%%%%%%%%%%%%%%%%%%%%%

\begin{abstract}
This document outlines the organization, architecture and goals of our
RoboCup@Home Domestic Standard Platform League team, \textit{team Oxford}(?). 
We are a new team that aspires to complete in international completions,
starting from 2018. Our research interests are centered around long term
autonomy, mobility, robot learning and knowledge representation. %This is
%supported by our research in
Advances in these directions will enable service robots to interact with humans
and complete useful everyday tasks in typical household settings.

\end{abstract}


%%%%%%%%%%%%%%%%%%%%%%%%%%%%%%%%%%%%%%%%%%%%%%%%%%%%%%%%%%%%%%%%%%%%%%%%%%%%%%%%%%%%

It must contain the following information:
\begin{itemize}
    \item Description of the approach planned to be implemented on the robot
    \item List of externally available components that are planned to be
    implemented (Open source software, web services, etc.)
    \item Innovative technology and scientific contribution
    \item Focus of research/research interests
    \item Re-usability of the system for other research groups
    \item Applicability of the approach in the real world
\end{itemize}
The Proposal should go into detail about the technical and scientific approach


%Here are some references that can be relevant to our story:
%\cite{havoutis13ijrr,Winkler2015,havoutis15clawar,Mastalli2015,Havoutis16SSRR,Zeestraten2017-RAL,Havoutis17ICRA,Zeestraten17IROS,Mastalli17ICRA}.

\section{Introduction}
\textit{Team Oxford} is a new team created within the Oxford Robotics Institute
(ORI) at the University of Oxford. The team consists of graduate students,
robotics researchers and faculty members of ORI.


The Domestic Standard Platform League affords a very tangible and 
interesting/challenging domain for our research work.

\section{Team Composition and Research Interests}

\section{Capabilities and Goals}
\begin{figure*}[!ht]
	\centering
	\subfigure{\resizebox{\textwidth}{!}{\includegraphics{images/baxter_learning_riemannian.png}}\label{fig:
all:a}}	
%	\vspace{-10pt}%
	\caption{Learning a skill from demonstration using Riemannian manifold representation. Covariance/correlation to optimal controller.}
	\label{fig:baxter_water_task}
\end{figure*}

\section{Conclusion}

\bibliographystyle{unsrt}
\bibliography{bibliography}

\end{document} 
