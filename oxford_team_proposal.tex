
%%%%%%%%%%%%%%%%%%%%%%% file typeinst.tex %%%%%%%%%%%%%%%%%%%%%%%%%
%
% This is the LaTeX source for the TDPTemplate using
% the LaTeX document class 'llncs.cls' Springer LNAI format
% used in the RoboCup Symposium submissions.
% http://www.springer.com/computer/lncs?SGWID=0-164-6-793341-0
%
% It may be used as a template for your own TDP - copy it
% to a new file with a new name and use it as the basis
% for your Team Description Paper
%
% NB: the document class 'llncs' has its own and detailed documentation, see
% ftp://ftp.springer.de/data/pubftp/pub/tex/latex/llncs/latex2e/llncsdoc.pdf
%
%%%%%%%%%%%%%%%%%%%%%%%%%%%%%%%%%%%%%%%%%%%%%%%%%%%%%%%%%%%%%%%%%%%

\documentclass[runningheads,a4paper]{llncs}
\usepackage{amssymb}
\setcounter{tocdepth}{3}
\usepackage{graphicx}
\usepackage{amssymb}
\usepackage[utf8]{inputenc}
\usepackage{url}
\usepackage{float}
\usepackage{amsmath}
\usepackage{graphicx}
\usepackage{wrapfig}

\usepackage{lipsum}
\newcommand{\BnL}[1][1em]{ \includegraphics[width=#1]{images/bnl.jpg} }

\begin{document}

\title{Lipsum 2017 Team Description Paper}

\author{Main-author \and Co-author \and Team Members }
\institute{Affiliation name and address, \\
\texttt{http://devoted-web-site.url}}
\maketitle


%%%%%%%%%%%%%%%%%%%%%%%%%%%%%%%%%%%%%%%%%%%%%%%%%%%%%%%%%%%%%%%%%%%%%%%%%%%%%%%%%%%%

\begin{abstract}

In your abstract, please state your main research line and your achievements of this year (on which problem or set of problems are you focusing all the team efforts). Tell why this research is important, how are you approaching to the problem solution and which results do you expect to obtain.

\end{abstract}


%%%%%%%%%%%%%%%%%%%%%%%%%%%%%%%%%%%%%%%%%%%%%%%%%%%%%%%%%%%%%%%%%%%%%%%%%%%%%%%%%%%%

\section{Introduction}


It must contain the following information:
\begin{itemize}
    \item Description of the approach planned to be implemented on the robot
    \item List of externally available components that are planned to be implemented (Open source software, web services, etc.)
    \item Innovative technology and scientific contribution
    \item Focus of research/research interests
    \item Re-usability of the system for other research groups
    \item Applicability of the approach in the real world
\end{itemize}
The Proposal should go into detail about the technical and scientific approach


Here are some references that can be relevant to our story:
\cite{havoutis13ijrr,Winkler2015,havoutis15clawar,Mastalli2015,Havoutis16SSRR,Zeestraten2017-RAL,Havoutis17ICRA,Zeestraten17IROS,Mastalli17ICRA}.
 

\bibliographystyle{unsrt}
\bibliography{bibliography}

\end{document} 
